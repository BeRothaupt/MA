% !TeX root = ../../../Main.tex
\chapter{Dynamic Programming}
\label{chapter3}

\begin{figure}[h]
	\includegraphics[width=\textwidth]{src/pics/dummy.jpg}
	\caption{Classification of Machine Learning algorithms}
	\label{fig:model-based-model-free} 
\end{figure}

This thesis deals with Dynamic Programming for policy optimization. Markus Zuern very successfully applied Trust Region Policy Optimization (TRPO) and Asynchronous Advantage Actor Critic (A3C) to a 2D trajectory optimization problem\cite{Zuern2017}. In his work, both algorithms deal with the complete environment, i.e. the glider dynamics and the stochastic wind distribution, at the same time. Whereas the wind distribution is unknown a priori, the glider dynamics are well known. Therefore, a DP algorithm can be applied to any scenario with calm air. The resulting optimal policy $\pi^*_{\text{calm}}$ (todo: dies überall übernehmen) then can be used as a starting point for TRPO in scenarios with unknown wind conditions.

\begin{figure}[h]
	\includegraphics[width=\textwidth]{src/pics/dummy.jpg}
	\caption{Figure 3.1 von Markus}
	\label{fig:RLmethods} 
\end{figure}

The term Dynamic Programming is used for optimization problems that can be decomposed into subproblems and solved by composing the solutions of those subproblems. If the subproblems are very similar, the solution of one subproblem can be used to calculate another one. In such cases, dynamic programming is applicable. In MDPs, these subproblems are similar in the sense that the Bellman Expectation Equation and the Bellman optimality equation hold true in all states, i.e. each state is connected to its successor state(s) by one of the Bellman Equations.

Also, the optimal action is found in all states the same way, as can be seen in section \ref{sec:PI}. If the algorithm is successful, the policy is optimal according to section \ref{sec:optimality}. Unlike with TRPO, optimization is not done through sampling complete trajectories, but by iterating the state-values and policy for each state independently. After one sweep over the state space, the value function and policy are updated and the process is repeated.\bigbreak

In theory, starting TRPO or A3C with a policy that is already optimal with respect to the scenario in calm air should accelerate training with updrafts because the information about how to behave in a situation with no wind is already "saved" in the policy.

\section{Approximate Dynamic Programming}

(hier noch einfügen, wie DP mit RL zusammenhängt)

\section{Reward Function}
\label{sec:reward}

As shown in Fig. \ref{fig:agent_env_system}, the agent interacts with his environment by, at each time step, picking an action $a_t$ and receiving an observation $o_t$ and a reward $r_t$. In RL, the reward function is the only means to give the agent information about what he should and should not do.

The reward function in this work is as follows:

\begin{equation}
\mathcal{R}(s,a): \mathcal{S} \mapsto \mathbb{R}: r_t =
\begin{cases}
-\Delta t & \text{if } s_{t+1} \neq s_{T+}\\
-\Delta t + \frac{d}{10} & \text{if } s_{t+1} = s_{T+}
\end{cases}
\label{eq:reward_function}
\end{equation}

(todo: abbildung dazu?)

where $s_{T+}$ is a terminal state that is also a target state. There are terminal states $s_{T-}$ where the agent is not meant to go. These terminal states are those where the agent touches the ground before reaching his target range. If the agent reaches his goal, he receives a final reward $r_T$ according to the second line of Eq. \ref{eq:reward_function}. 

It is possible to penalize undesired behavior of the agent by defining a large negative reward if, for example, the agent touches the ground before reaching the target distance or flies too far in the wrong direction. This is, however, not necessary, if the final reward is sufficiently high. If the agent has to cover a distance of 500 m, he receives a final reward of +50. with a mean ground speed of about $15-20\frac{\text{m}}{\text{s}}$, covering 500 m takes between 25 and 33.3 s. If the agent gets $-1$ reward per second it takes him to reach 500 m, the expected return from a state far away from the goal can still get up to about $33.3\cdot (-1) +50=16.7$. This would not work if the final reward was only $+\frac{\text{d}}{20}$. The important thing here is, that reaching a target state must always yield a higher return than ending an episode on the ground from any state $s \in \mathcal{S}$.


\section{The Principle of Optimality}
\label{sec:optimality}

According to Bellman's principle of optimality, the solution of some optimization problems can be put together from the solution of subproblems. This is the basis for all dynamic programming algorithms. If an agent chooses the optimal action in a state $s_t$ and all states it visits afterwards ($s_k$, $k=t+1, t+2, ...,T$) , the resulting trajectory is the optimal one from $s_t$ to the terminal state $s_T$.

The term \textit{principle of optimality} was first used by Richard E. Bellman in 1954 \cite{Bellman1954}. He proved that various optimization problems could be solved efficiently by making use of the principle of optimality instead of traditional optimal control theory. In his book \textit{Applied Dynamic Programming}, Bellman covered numerous examples of optimal control problems and shows how they can be solved by Dynamic Programming \cite{Bellman1962}.

\section{Types of Dynamic Programming Algorithms}

\subsection{Policy Evaluation}
\label{subsection:policy_evaluation}

In Policy Evaluation, the goal is to find an approximation of the value function $V_\pi$ that corresponds to a given policy $\pi$. The value of a state is defined as the expected total return from that state onwards until the episode terminates. Obviously, the expectation in equation \ref{eq:bellman_exp} depends on the actions that are taken at each step and therefore on the policy that is evaluated. The state value that corresponds to a specific policy $\pi$ is denoted by $V_\pi(s)$.

\begin{align}
V_\pi(s_t)&=\mathbb{E}[G_t|s_t=s]\\ &= \mathbb{E}\left[ \sum_{k=0}^{T-t-1}\gamma^k r_{t+k+1}|s_t=s\right] \\
&=\mathbb{E}[r_{t+1}+\gamma^1 r_{t+2}+\gamma^2 r_{t+3}+\gamma^3 r_{t+4}+...+\gamma^{T-t-1}r_T|s_t=s]
\label{eq:bellman_exp}
\end{align}

With a stochastic discrete policy, equation \ref{eq:bellman_exp} becomes

\begin{equation}
V_\pi(s_t)=\sum_{a}\pi(a|s)(\mathcal{R}(s,a)+\gamma V_\pi(s'))
\label{eq:bellman_exp_discrete_policy}
\end{equation}

Recall the definition of a state value $V(s) = \mathbb{E}[G_t|s_t=s]$ from Eq. \ref{eq:state_value_fun}. This equation can be used as a mapping to update the value of $s_t$ with the expected return according to the current policy.

\begin{equation}
V_\pi(s_t) \mathrel{\reflectbox{\ensuremath{\mapsto}}} \mathbb{E}[ \sum_{k=0}^{T-t-1}\gamma^k r_{t+k+1}|s_t=s]
\label{eq:bellman_exp_update}
\end{equation}

In Eq. \ref{eq:bellman_exp}, all terms expect the first one can be replaced by the expected return from $s_{t+1}$ onwards:

\begin{align}
V_\pi(s_t) \mathrel{\reflectbox{\ensuremath{\mapsto}}} 
&=\mathbb{E}[r_{t+1}+\gamma G_{t+1}|s_t=s]\\
&=r_{t+1}+\gamma V_\pi(s_{t+1})
\label{eq:bellman_exp_update_bootstrapped}
\end{align}

With a discrete stochastic policy, equation \ref{eq:bellman_exp_update_bootstrapped} becomes


\begin{equation}
V_{\pi}(s_t) \mathrel{\reflectbox{\ensuremath{\mapsto}}} \sum_{a}\pi(a|s)(r_t+\gamma V_\pi(s'))
\label{eq:bellman_exp_update_discrete}
\end{equation}

In a state space with a finite set of states $\mathcal{S}$, writing down Eq. \ref{eq:bellman_exp_update_discrete} for each state results in a system of $|\mathcal{S}|$ linear equations that can be solved for $V(s)$. If the number of states is very large, this is not very efficient. 

Instead, \ref{eq:bellman_exp} is usually solved iteratively. The simplest way is to perform a one step lookahead from each state to get $r_t$ and $V(s_{t+t})$ from the current estimate of the state value function. Once all values are calculated, the values of all states are updated simultaneously. Unlike most RL-algorithms, the state values are not updated along a trajectory. Instead, each state is updated independently.\footnote{There are variants of Policy Iteration, where the updates are performed asynchronously. This means that, for example, all neighbors of a terminal state are updated first. After that, the states, that lie next to these states, are updated, and so on. They usually converge faster, but at the expense of implementation complexity.}

Each set of state values approximates the true value function of the problem better than the previous set of estimates. Every state value converges to the true state value under the given policy. It is not efficient to wait until the values have fully converged, i.e. the true value function $V_\pi$ is reached. Instead the process is repeated until the maximum absolute change in values lies beneath a certain threshold $\epsilon_V$. This maximum absolute change can be expressed by the supremum-norm $||(\cdot)||_\infty$ (c.f. section \ref{sec:contraction_mappings}). Algorithm \ref{algo:pe} shows a possible implementation of iterative policy evaluation.

\begin{equation}
||V_{k}-V_{k-1}||_\infty<\epsilon_V
\label{eq:pe_stopping_criterion}
\end{equation}

\begin{algorithm}[hbt]
	\caption{Iterative Policy Evaluation}
	\begin{algorithmic}[0] % 1 to show code line numbers
		
		\Function{Policy Evaluation}{}
			\State Initialize $V(s) = 0 \; \forall \; s \in \mathcal{S}$ and load policy $\pi$.
			\While {true}
				\ForAll{$s \in \mathcal{S}$}
					\State $V_{\pi,new}(s) \mathrel{\reflectbox{\ensuremath{\mapsto}}} r+\gamma V_ {\pi,old}(s')$
				\EndFor
				\State update value function
				\If {$||V_{new}-V_{old}||_\infty \leq \epsilon_V$}
					\State break
				\EndIf
			\EndWhile
		\EndFunction
	\end{algorithmic}
	\label{algo:pe}
\end{algorithm}

\subsection{Policy Iteration}
\label{sec:PI}
Policy iteration is an iterative way to calculate an estimate of the optimal policy of an MDP. Starting with an arbitrary policy and value function, one iteration of Policy Evaluation is performed to get an estimate of $V_\pi$. After that, a new policy is generated that chooses the greedy action with respect to $V_\pi$ in each state. This alternating process of Policy Evaluation and Policy Improvement is repeated until the policy is satisfactory.

\begin{equation}
a_{\text{greedy}}(s_t) = \underset{a}{\text{argmax}}[Q(s_t,a_t)]
\end{equation}

For a deterministic, greedy policy $\pi_{\text{greedy}}$, this yields:

\begin{align}
\pi(a_{\text{greedy}}|s_t)&=\mathbb{P}[a_t=a_{\text{greedy}}(s_t)|s_t] \\ &=1
\end{align}

and for a stochastic policy $\pi$:

\begin{equation}
\mathbb{E}[\pi(a_t | s_t)] \mathrel{\reflectbox{\ensuremath{\mapsto}}} a_{\text{greedy}}
\end{equation}

As mentioned in subsection \ref{subsection:policy_evaluation}, each policy evaluation step usually ends if the maximum difference between two subsequent values of the same state are sufficiently close.

There is a version of Policy Iteration where only one value update is done before updating the policy. This algorithm is called Optimistic Policy Iteration (OPI). Although the first estimate of the value function is only a rough approximation, OPI also converges to the optimal value function and policy. Equation \ref{eq:pi_scheme} illustrates the policy iteration scheme.

\begin{equation*}
\pi_0 \overset{evaluate}{\longrightarrow} V_1 \overset{improve}{\longrightarrow} \pi_1 \overset{evaluate}{\longrightarrow} V_2 \overset{improve}{\longrightarrow} \pi_2 \overset{evaluate}{\longrightarrow} ... \overset{evaluate}{\longrightarrow} V_* \overset{improve}{\longrightarrow} \pi_*
\label{eq:pi_scheme}
\end{equation*}

Another way to picture Policy Iteration is shown in figure \ref{fig:PI_triangle}. Every PI-step brings the Value Function $V(s)$ closer to $V_*(s)$ and the policy $\pi$ closer to $\pi_*$.

\begin{figure}[h]
	\includegraphics[width=\textwidth]{src/pics/dummy.jpg}
	\caption{The Policy Iteration Algorithm}
	\label{fig:PI_triangle} 
\end{figure}

The performance of both algorithms on the glider-problem is compared in chapter \ref{chapter6}. \bigbreak

In a discrete MDP, there is a finite number of states and actions. At each state $s \in \mathcal{S}$, the agent can choose the action $a$ from a set $\mathcal{A}$ of possible actions. In such a scenario, there exist $|\mathcal{A}|^{|\mathcal{S}|}$ different policies. It therefore takes $|\mathcal{A}|^{|\mathcal{S}|}$ iterations to find the optimal policy assuming no policy occurs twice. If that was the case, this policy would have to occur in two subsequent iterations and therefore be the fixed point $\pi_*$ of the policy sequence generated by PI.

As the results in chapter \ref{chapter6} show, PI and VI converge much faster in practice than the upper bound $|\mathcal{A}|^{|\mathcal{S}|}$ may suggest. Algorithm \ref{algo:gpi} shows one way to implement generalized policy iteration. The pseudocode for optimistic policy iteration can be found in algorithm \ref{algo:opi} in appendix \ref{appendix_D}.

\begin{algorithm}[hbt]
	\caption{Generalized Policy Iteration}
	\begin{algorithmic}[0] % 1 to show code line numbers
		\Function{Generalized Policy Iteration}{}
			\State Initialize $V(s) = 0 \; \forall \; s \in \mathcal{S}$
			\State Load arbitrary initial policy $\pi_0$.
			\State Initialize $m$,$\epsilon_V$
			\While {true}			
			\Function{Policy Evaluation}{}
				\While {true}
				\ForAll{$s \in \mathcal{S}$}
					\State $V_{\pi,new}(s) \mathrel{\reflectbox{\ensuremath{\mapsto}}} r+\gamma V_ {\pi,old}(s')$
				\EndFor
				\State update value function
				\If {$||V_{new}-V_{old}||_\infty \leq \epsilon_V$}
					\State break
				\EndIf
				\EndWhile
			\EndFunction
			\Function{Policy Improvement}{}
				\ForAll{$s \in \mathcal{S}$}
					\State sample $m$ actions $a_n(s)$
					\ForAll {$a_n$}
						\State $Q(s,a_n) = r + \gamma V(s')$
					\EndFor
					\State $a_{greedy}(s)=\underset{a_n}{\text{argmax}}[Q(s,a_n)]$
				\EndFor
			\EndFunction
			\State
			\State train policy on $a_{greedy,new}$ to obtain $\pi_{new}$
			\If {$||a_{greedy,new}-a_{greedy,old}||_\infty = 0$}
				\State break
			\EndIf

			\EndWhile
			\State
		\EndFunction
	\end{algorithmic}
	\label{algo:gpi}
\end{algorithm}

\subsection{Value Iteration}
\label{subsection:VI}
Value iteration is similar to policy iteration, but instead of calculating a new policy at each iteration step, the state values $V(s)$ are directly updated with the maximum possible successor value that is achievable starting from $s$. This yields a sequence of value functions, each one being a better estimate of $V_*$ than its predecessor. Each of the intermediate value functions is abstract in so far as there does not have to exist an explicit policy corresponding to it (c.f. \cite[lecture~3]{Silver2015}. In a possibly discounted MDP with a deterministic environment, the update rule for each state value can be seen in equation \ref{eq:value_iteration_update}.

\begin{equation}
V(s_t) \mathrel{\reflectbox{\ensuremath{\mapsto}}} 
\underset{a}{\text{max}}[r_{t+1}+\gamma V(s_{t+1})|s_t=s]
\label{eq:value_iteration_update}
\end{equation}

In principle, this is equivalent to OPI, where after one state value update at each state, the policy is replaced by the greedy policy with respect to the new value function. The only difference is that value iteration does not output an intermediate policy at each Iteration. Instead, it only iterates value functions. 

\begin{equation*}
V_1 \longrightarrow V_2 \longrightarrow V_3 \longrightarrow ... \longrightarrow  V_*
\label{eq:vi_scheme}
\end{equation*}

In practice, a stopping criterion is introduced to bound calculation time. In this work, iteration is stopped if $||V_{k}-V_{k-1}||_\infty<\epsilon_V$, identical to equation \ref{eq:pe_stopping_criterion}. The last iterate $V_n$ is considered an approximation of $V_*$. If $V_n$ is close enough to the true optimal value function $V_*$, the (approximately) optimal policy can be calculated in the end by acting greedily with respect to $V_n$. Recall that, if Value Iteration is stopped too early, there might not even be a real policy that corresponds to the last value function iterate $V_n$ \cite[lecture~3]{Silver2015}. A possible implementation of value iteration is shown in algorithm \ref{algo:vi}.

\begin{algorithm}[hbt]
	\caption{Value Iteration}
	\begin{algorithmic}[0] % 1 to show code line numbers
		\Function{Value Iteration}{}
			\State Initialize $V(s) = 0 \; \forall \; s \in \mathcal{S}$
			\State Initialize $m$,\;$\epsilon_V$
			\While {true}	
				\ForAll {$s \in \mathcal{S}$}
					\State sample $m$ actions $a_n(s)$
					\ForAll {$a_n$}
						\State $Q(s,a_n) = r + \gamma V(s')$
					\EndFor		
					\State $V_{new}(s)=\underset{a_n}{\text{max}}[Q(s,a_n)]$
				\EndFor
				\If {$||V_{new}-V_{old}||_\infty \leq \epsilon_V$}
					\State break
				\EndIf
			\EndWhile
			\Function{Policy Improvement}{}
				\ForAll{$s \in \mathcal{S}$}
					\State sample $m$ actions $a_n(s)$
					\ForAll {$a_n$}
						\State $Q(s,a_n) = r + \gamma V(s')$
					\EndFor
					\State $a_{greedy}(s)=\underset{a_n}{\text{argmax}}[Q(s,a_n)]$
				\EndFor
			\EndFunction
			\State train policy on the latest $a_{greedy,new}$ to obtain an estimate of $\pi_*$
		\EndFunction
	\end{algorithmic}
	\label{algo:vi}
\end{algorithm}

\section{The Contraction Mapping Theorem}
\label{sec:contraction_mappings}
A contraction mapping $f: M \to M$ on a metric space $M$ has the following property:
\begin{equation}
|f(x_1)-f(x_2)| \leq \gamma |x_1-x_2|
\end{equation}

with $x_1,x_2 \in M$ and $\gamma \in [0,1)$. In this context,  $|(\cdot)|$ denotes an arbitrary metric on $M$.

The term contraction reflects the fact that, graphically speaking, a contraction mapping reduces the distance between $x_1$ and $x_2$.

\subsubsection{Discounted MDPs}

Policy Evaluation in discounted MDPs is a contraction mapping. This result can be obtained in few steps. The mapping in equation \ref{eq:bellman_exp_update_discrete} and \ref{eq:bellman_operator_discounted} is known as the \textit{Bellman Operator}. It maps from a state value function to a state value function. If the Bellman Operator is applied multiple times, this can be expressed by $T^n(v)$.

\begin{equation}
T_\pi(v(s'))=\mathcal{R}_\pi+\gamma \mathcal{P}_\pi v(s')
\label{eq:bellman_operator_discounted}
\end{equation}

If the Bellman Operator is applied to two state value functions $V_1(s)$ and $V_2(s)$, it reduces the distance between the two in value function space. This distance can be expressed by the supremum norm which is defined as follows:

\begin{equation}
||V||_\infty = \max_{s \in \mathcal{S}} |V(s)|
\end{equation}

The supremum norm of $V_1 - V_2$ is therefore

\begin{equation}
||V_1(s) - V_2(s)||_\infty = \max_{s \in \mathcal{S}} |V_1(s)-V_2(s)|
\end{equation}

which is the biggest absolute difference of values of the same state in the state space. If the Bellman Operator is applied to both $V_1$ and $V_2$, the contraction property proof is straightforward. For simplicity, the arguments of $V_1(s)$ and $V_2(s)$ are omitted in the following equations.
\begin{align}
||T_\pi(V_1)-T_\pi(V_2)||_\infty &= ||(\mathcal{R}_\pi+\gamma \mathcal{P}_\pi V_1)-(\mathcal{R}_\pi+\gamma \mathcal{P}_\pi V_2)||_\infty \label{eq:4.15}\\
&=||\gamma\; \mathcal{P}_\pi\;(V_1 - V_2)||_\infty \label{eq:4.16}\\
&\leq ||\gamma \mathcal{P}_\pi ||V_1 - V_2||_\infty ||_\infty \label{eq:4.17} \\
&\leq ||\gamma ||V_1 - V_2||_\infty ||_\infty \label{eq:4.18}\\
&=\gamma ||V_1 - V_2||_\infty \label{eq:4.19}
\end{align} 

Whenever $s'$ happens to be the state where $V_1$ and $V_2$ differ the most, equality holds in line \ref{eq:4.17}. If not, applying the supremum norm to $V_1(s')-V_2(s')$ increases the value of the expression. $\mathcal{P}^\pi$ must always be smaller than or equal to one. Similar to line \ref{eq:4.17}, equality in line \ref{eq:4.18} holds only if $\mathcal{P}^\pi=1$. Replacing $\mathcal{P}^\pi$ by one in line \ref{eq:4.17} yields line \ref{eq:4.18}. The left side of line \ref{eq:4.15} must thus be less than line \ref{eq:4.19} which proves the contraction property of the Bellman Operator in discounted MDPs.

\subsubsection{Undiscounted MDPs}

An MDP, where the discount factor $\gamma$ is one, is called an undiscounted MDP. In such an MDP, the Bellman Operator looks like in equation \ref{eq:bellman_undiscounted}.

\begin{equation}
T_\pi(v(s))=\mathcal{R}_\pi + \mathcal{P}_\pi V(s')
\label{eq:bellman_undiscounted}
\end{equation}

If $\gamma$ is one, the above proof of the contraction property does not hold anymore. It can however be proven, that the Bellman Operator still contracts. In the undiscounted case, this takes more than one step. Equation \ref{eq:n-step-contraction} shows the n-step contraction property. For simplicity, the index $\pi$ and the argument $s$ are omitted.

\begin{equation}
||T^{(n)}(V_1)-T^{(n)}(V_2)||_\infty \leq ||V_1-V_2||_\infty
\label{eq:n-step-contraction}
\end{equation}

Equation \ref{eq:n-step-contraction} means that applying the Bellman Operator n times to two value functions brings them closer together in value function space. As a result, the Bellman Operator also leads to $V_1 \approx V_2$ eventually in undiscounted MDPs.

A proof that policy iteration and value iteration also converge in undiscounted MDPs can be found in \cite{},\cite{} and \cite{}.

